\documentclass[12pt]{article}
\usepackage[a4paper, total={6in, 10in}]{geometry}


\usepackage{enumitem}   
\usepackage{url}
\usepackage{graphicx}
\graphicspath{{figs/}} 


\title{
Quantification of Dynamic Facial Expressions \\
with Shannon Entropy
in Human-Humanoid Interaction
} 
\author{Miguel Xochicale \\
School of Engineering \\
University of Birmingham}


\date{13th January 2019}

\begin{document}
\maketitle
\thispagestyle{empty} %No number

Research on understanding and quantifying movement variability 
with nonlinear analyses has been well established in the last three 
decades in areas such as biomechanics, sport science, psychology, 
cognitive science, and neuroscience \cite{davids2003}.
This work is hypothesising that nonlinear analyses 
can be used to quantify subtle variations of facial expressions 
that can be related to different mental states 
(i.e. anxiety, disinterest, relief, etc) \cite{back2014}.
This hypothesis has then led the author to ask two research questions: 
\begin{enumerate}[label=(\roman*)]
\item how the quantification of facial expressions can be related to  
	the complexity of facial expressions?, and 
\item does the quantification of the complexity for facial expressions 
	can tell us something about the state of mind of a person?, 
\end{enumerate}


In order to give insights into the raised questions,
this work is proposing 
the use of Recurrence Quantification Analysis (RQA) 
to quantify the complexity of facial expressions 
which is based on previous investigations of the author 
with nonlinear dynamics to quantify movement variability 
in human-humanoid interaction \cite{XochicalePhDThesis2018}.
RQA computes measurements based on the density of 
recurrence points of diagonal line 
structures in the Recurrence Plots.
For which, RQA provide understanding of the dynamics of a system i.e. 
the determinism (predictability of a system) or 
Shannon entropy (complexity of a system) \cite{marwan2007}.
With that in mind, a pilot experiment is designed 
to show the complexity of facial expressions variability. 
In the experiment one participant (the author) 
were asked to perform 
three levels of variability of face expressions:
(i) neutral variations, (ii) slow variations, and (iii) faster variations.
Then, using time-series data of 67 face landmarks collected 
with OpenFace \cite{baltrusaitis2018}, 3D plots
of RQA Entr (Shannon entropy) were computed in order
to quantify the complexity of face expressions and 
therefore relate 3D plots of RQA Entr 
to both (i) the subtle variations of facial expressions 
and (ii) the state of mind of a person. Additionally,
this work will present potential applications 
in the context of human-humanoid interaction
for automatic quantification of face expressions 
that can be related to person's state of mind. 


\bibliographystyle{alpha}
\bibliography{references/references}


\end{document}
