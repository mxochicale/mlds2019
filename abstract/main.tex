\documentclass[12pt]{article}
\usepackage[a4paper, total={6in, 10in}]{geometry}
%\usepackage[top=0in, bottom=1in, left=1.25in, right=1.25in]{geometry}


\usepackage{enumitem}   
\usepackage{url}
\usepackage{graphicx}
\graphicspath{{figs/}} 


\title{
Quantification of Complexity for \\
Facial Expressions with Shannon Entropy
} 
\author{Miguel Xochicale \\
School of Engineering \\
University of Birmingham}


\date{13th January 2019}

\begin{document}
\maketitle
%\thispagestyle{empty} %No number


%\begin{abstract}
%...
%\end{abstract}


Research on measurement and understanding of movement variability 
with nonlinear analyses has been well established in the last three 
decades in areas of biomechanics, sport science, psychology, cognitive science,
and neuroscience \cite{davids2003}.
Considering nonlinear analyses to quantify complexity of 
facial expressions,
this work is hypothesising that such approach can be used to measure 
subtle variations of facial expressions in order to quantify 
different mental states (i.e. anxiety, disinterest, relief, etc) \cite{back2014}.
Such hypothesis has then led the author to ask two research questions: 
\begin{enumerate}[label=(\roman*)]
\item how the quantification of facial expressions can be related to  
	the complexity of facial expressions?.
\item does the quantification of the complexity of facial expressions 
	can tell us something about the state of mind of a person?, 
\end{enumerate}

Hence, with the investigation of nonlinear dynamics to 
quantify movement variability in human-humanoid interaction 
\cite{XochicalePhDThesis2018}, this work is proposing 
the use of Recurrence Quantification Analysis (RQA) 
to quantify the complexity of facial expressions to give 
insights into the raised questions.
RQA computes measurements based on the density of 
recurrence points of diagonal or vertical line 
structures in Recurrence Plots to provide understanding 
of the dynamics of a system i.e. 
the determinism (predictability) or 
Shannon entropy (complexity) \cite{marwan2007}.
Hence, this work presents a pilot experiment of one participant (the author) 
who were asked to perform three levels of 
face expressions:
(i) neutral variation, (ii) slow variation and (iii) faster variations
to show the complexity of facial expressions variability.
Then, using 67 face landmarks time-series data collected 
with OpenFace \cite{baltrusaitis2018}, this work shows 3D plots
of RQA Entr (Shannon entropy) that can be used to quantify the 
complexity of face expressions and therefore relate such metric 
to both (i) subtle variations of facial expressions 
and (ii) the state of mind of a person. Additionally,
this work presents applications of the proposed
approach in the context of human-humanoid interaction. 

\bibliographystyle{alpha}
\bibliography{references/references}


\end{document}
